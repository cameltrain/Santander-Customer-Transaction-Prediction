
% Default to the notebook output style

    


% Inherit from the specified cell style.




    
\documentclass[11pt]{article}

    
    
    \usepackage[T1]{fontenc}
    % Nicer default font (+ math font) than Computer Modern for most use cases
    \usepackage{mathpazo}

    % Basic figure setup, for now with no caption control since it's done
    % automatically by Pandoc (which extracts ![](path) syntax from Markdown).
    \usepackage{graphicx}
    % We will generate all images so they have a width \maxwidth. This means
    % that they will get their normal width if they fit onto the page, but
    % are scaled down if they would overflow the margins.
    \makeatletter
    \def\maxwidth{\ifdim\Gin@nat@width>\linewidth\linewidth
    \else\Gin@nat@width\fi}
    \makeatother
    \let\Oldincludegraphics\includegraphics
    % Set max figure width to be 80% of text width, for now hardcoded.
    \renewcommand{\includegraphics}[1]{\Oldincludegraphics[width=.8\maxwidth]{#1}}
    % Ensure that by default, figures have no caption (until we provide a
    % proper Figure object with a Caption API and a way to capture that
    % in the conversion process - todo).
    \usepackage{caption}
    \DeclareCaptionLabelFormat{nolabel}{}
    \captionsetup{labelformat=nolabel}

    \usepackage{adjustbox} % Used to constrain images to a maximum size 
    \usepackage{xcolor} % Allow colors to be defined
    \usepackage{enumerate} % Needed for markdown enumerations to work
    \usepackage{geometry} % Used to adjust the document margins
    \usepackage{amsmath} % Equations
    \usepackage{amssymb} % Equations
    \usepackage{textcomp} % defines textquotesingle
    % Hack from http://tex.stackexchange.com/a/47451/13684:
    \AtBeginDocument{%
        \def\PYZsq{\textquotesingle}% Upright quotes in Pygmentized code
    }
    \usepackage{upquote} % Upright quotes for verbatim code
    \usepackage{eurosym} % defines \euro
    \usepackage[mathletters]{ucs} % Extended unicode (utf-8) support
    \usepackage[utf8x]{inputenc} % Allow utf-8 characters in the tex document
    \usepackage{fancyvrb} % verbatim replacement that allows latex
    \usepackage{grffile} % extends the file name processing of package graphics 
                         % to support a larger range 
    % The hyperref package gives us a pdf with properly built
    % internal navigation ('pdf bookmarks' for the table of contents,
    % internal cross-reference links, web links for URLs, etc.)
    \usepackage{hyperref}
    \usepackage{longtable} % longtable support required by pandoc >1.10
    \usepackage{booktabs}  % table support for pandoc > 1.12.2
    \usepackage[inline]{enumitem} % IRkernel/repr support (it uses the enumerate* environment)
    \usepackage[normalem]{ulem} % ulem is needed to support strikethroughs (\sout)
                                % normalem makes italics be italics, not underlines
    \usepackage{mathrsfs}
    

    
    
    % Colors for the hyperref package
    \definecolor{urlcolor}{rgb}{0,.145,.698}
    \definecolor{linkcolor}{rgb}{.71,0.21,0.01}
    \definecolor{citecolor}{rgb}{.12,.54,.11}

    % ANSI colors
    \definecolor{ansi-black}{HTML}{3E424D}
    \definecolor{ansi-black-intense}{HTML}{282C36}
    \definecolor{ansi-red}{HTML}{E75C58}
    \definecolor{ansi-red-intense}{HTML}{B22B31}
    \definecolor{ansi-green}{HTML}{00A250}
    \definecolor{ansi-green-intense}{HTML}{007427}
    \definecolor{ansi-yellow}{HTML}{DDB62B}
    \definecolor{ansi-yellow-intense}{HTML}{B27D12}
    \definecolor{ansi-blue}{HTML}{208FFB}
    \definecolor{ansi-blue-intense}{HTML}{0065CA}
    \definecolor{ansi-magenta}{HTML}{D160C4}
    \definecolor{ansi-magenta-intense}{HTML}{A03196}
    \definecolor{ansi-cyan}{HTML}{60C6C8}
    \definecolor{ansi-cyan-intense}{HTML}{258F8F}
    \definecolor{ansi-white}{HTML}{C5C1B4}
    \definecolor{ansi-white-intense}{HTML}{A1A6B2}
    \definecolor{ansi-default-inverse-fg}{HTML}{FFFFFF}
    \definecolor{ansi-default-inverse-bg}{HTML}{000000}

    % commands and environments needed by pandoc snippets
    % extracted from the output of `pandoc -s`
    \providecommand{\tightlist}{%
      \setlength{\itemsep}{0pt}\setlength{\parskip}{0pt}}
    \DefineVerbatimEnvironment{Highlighting}{Verbatim}{commandchars=\\\{\}}
    % Add ',fontsize=\small' for more characters per line
    \newenvironment{Shaded}{}{}
    \newcommand{\KeywordTok}[1]{\textcolor[rgb]{0.00,0.44,0.13}{\textbf{{#1}}}}
    \newcommand{\DataTypeTok}[1]{\textcolor[rgb]{0.56,0.13,0.00}{{#1}}}
    \newcommand{\DecValTok}[1]{\textcolor[rgb]{0.25,0.63,0.44}{{#1}}}
    \newcommand{\BaseNTok}[1]{\textcolor[rgb]{0.25,0.63,0.44}{{#1}}}
    \newcommand{\FloatTok}[1]{\textcolor[rgb]{0.25,0.63,0.44}{{#1}}}
    \newcommand{\CharTok}[1]{\textcolor[rgb]{0.25,0.44,0.63}{{#1}}}
    \newcommand{\StringTok}[1]{\textcolor[rgb]{0.25,0.44,0.63}{{#1}}}
    \newcommand{\CommentTok}[1]{\textcolor[rgb]{0.38,0.63,0.69}{\textit{{#1}}}}
    \newcommand{\OtherTok}[1]{\textcolor[rgb]{0.00,0.44,0.13}{{#1}}}
    \newcommand{\AlertTok}[1]{\textcolor[rgb]{1.00,0.00,0.00}{\textbf{{#1}}}}
    \newcommand{\FunctionTok}[1]{\textcolor[rgb]{0.02,0.16,0.49}{{#1}}}
    \newcommand{\RegionMarkerTok}[1]{{#1}}
    \newcommand{\ErrorTok}[1]{\textcolor[rgb]{1.00,0.00,0.00}{\textbf{{#1}}}}
    \newcommand{\NormalTok}[1]{{#1}}
    
    % Additional commands for more recent versions of Pandoc
    \newcommand{\ConstantTok}[1]{\textcolor[rgb]{0.53,0.00,0.00}{{#1}}}
    \newcommand{\SpecialCharTok}[1]{\textcolor[rgb]{0.25,0.44,0.63}{{#1}}}
    \newcommand{\VerbatimStringTok}[1]{\textcolor[rgb]{0.25,0.44,0.63}{{#1}}}
    \newcommand{\SpecialStringTok}[1]{\textcolor[rgb]{0.73,0.40,0.53}{{#1}}}
    \newcommand{\ImportTok}[1]{{#1}}
    \newcommand{\DocumentationTok}[1]{\textcolor[rgb]{0.73,0.13,0.13}{\textit{{#1}}}}
    \newcommand{\AnnotationTok}[1]{\textcolor[rgb]{0.38,0.63,0.69}{\textbf{\textit{{#1}}}}}
    \newcommand{\CommentVarTok}[1]{\textcolor[rgb]{0.38,0.63,0.69}{\textbf{\textit{{#1}}}}}
    \newcommand{\VariableTok}[1]{\textcolor[rgb]{0.10,0.09,0.49}{{#1}}}
    \newcommand{\ControlFlowTok}[1]{\textcolor[rgb]{0.00,0.44,0.13}{\textbf{{#1}}}}
    \newcommand{\OperatorTok}[1]{\textcolor[rgb]{0.40,0.40,0.40}{{#1}}}
    \newcommand{\BuiltInTok}[1]{{#1}}
    \newcommand{\ExtensionTok}[1]{{#1}}
    \newcommand{\PreprocessorTok}[1]{\textcolor[rgb]{0.74,0.48,0.00}{{#1}}}
    \newcommand{\AttributeTok}[1]{\textcolor[rgb]{0.49,0.56,0.16}{{#1}}}
    \newcommand{\InformationTok}[1]{\textcolor[rgb]{0.38,0.63,0.69}{\textbf{\textit{{#1}}}}}
    \newcommand{\WarningTok}[1]{\textcolor[rgb]{0.38,0.63,0.69}{\textbf{\textit{{#1}}}}}
    
    
    % Define a nice break command that doesn't care if a line doesn't already
    % exist.
    \def\br{\hspace*{\fill} \\* }
    % Math Jax compatibility definitions
    \def\gt{>}
    \def\lt{<}
    \let\Oldtex\TeX
    \let\Oldlatex\LaTeX
    \renewcommand{\TeX}{\textrm{\Oldtex}}
    \renewcommand{\LaTeX}{\textrm{\Oldlatex}}
    % Document parameters
    % Document title
    \title{Santander Customer Transaction Prediction 01}
    
    
    
    
    

    % Pygments definitions
    
\makeatletter
\def\PY@reset{\let\PY@it=\relax \let\PY@bf=\relax%
    \let\PY@ul=\relax \let\PY@tc=\relax%
    \let\PY@bc=\relax \let\PY@ff=\relax}
\def\PY@tok#1{\csname PY@tok@#1\endcsname}
\def\PY@toks#1+{\ifx\relax#1\empty\else%
    \PY@tok{#1}\expandafter\PY@toks\fi}
\def\PY@do#1{\PY@bc{\PY@tc{\PY@ul{%
    \PY@it{\PY@bf{\PY@ff{#1}}}}}}}
\def\PY#1#2{\PY@reset\PY@toks#1+\relax+\PY@do{#2}}

\expandafter\def\csname PY@tok@w\endcsname{\def\PY@tc##1{\textcolor[rgb]{0.73,0.73,0.73}{##1}}}
\expandafter\def\csname PY@tok@c\endcsname{\let\PY@it=\textit\def\PY@tc##1{\textcolor[rgb]{0.25,0.50,0.50}{##1}}}
\expandafter\def\csname PY@tok@cp\endcsname{\def\PY@tc##1{\textcolor[rgb]{0.74,0.48,0.00}{##1}}}
\expandafter\def\csname PY@tok@k\endcsname{\let\PY@bf=\textbf\def\PY@tc##1{\textcolor[rgb]{0.00,0.50,0.00}{##1}}}
\expandafter\def\csname PY@tok@kp\endcsname{\def\PY@tc##1{\textcolor[rgb]{0.00,0.50,0.00}{##1}}}
\expandafter\def\csname PY@tok@kt\endcsname{\def\PY@tc##1{\textcolor[rgb]{0.69,0.00,0.25}{##1}}}
\expandafter\def\csname PY@tok@o\endcsname{\def\PY@tc##1{\textcolor[rgb]{0.40,0.40,0.40}{##1}}}
\expandafter\def\csname PY@tok@ow\endcsname{\let\PY@bf=\textbf\def\PY@tc##1{\textcolor[rgb]{0.67,0.13,1.00}{##1}}}
\expandafter\def\csname PY@tok@nb\endcsname{\def\PY@tc##1{\textcolor[rgb]{0.00,0.50,0.00}{##1}}}
\expandafter\def\csname PY@tok@nf\endcsname{\def\PY@tc##1{\textcolor[rgb]{0.00,0.00,1.00}{##1}}}
\expandafter\def\csname PY@tok@nc\endcsname{\let\PY@bf=\textbf\def\PY@tc##1{\textcolor[rgb]{0.00,0.00,1.00}{##1}}}
\expandafter\def\csname PY@tok@nn\endcsname{\let\PY@bf=\textbf\def\PY@tc##1{\textcolor[rgb]{0.00,0.00,1.00}{##1}}}
\expandafter\def\csname PY@tok@ne\endcsname{\let\PY@bf=\textbf\def\PY@tc##1{\textcolor[rgb]{0.82,0.25,0.23}{##1}}}
\expandafter\def\csname PY@tok@nv\endcsname{\def\PY@tc##1{\textcolor[rgb]{0.10,0.09,0.49}{##1}}}
\expandafter\def\csname PY@tok@no\endcsname{\def\PY@tc##1{\textcolor[rgb]{0.53,0.00,0.00}{##1}}}
\expandafter\def\csname PY@tok@nl\endcsname{\def\PY@tc##1{\textcolor[rgb]{0.63,0.63,0.00}{##1}}}
\expandafter\def\csname PY@tok@ni\endcsname{\let\PY@bf=\textbf\def\PY@tc##1{\textcolor[rgb]{0.60,0.60,0.60}{##1}}}
\expandafter\def\csname PY@tok@na\endcsname{\def\PY@tc##1{\textcolor[rgb]{0.49,0.56,0.16}{##1}}}
\expandafter\def\csname PY@tok@nt\endcsname{\let\PY@bf=\textbf\def\PY@tc##1{\textcolor[rgb]{0.00,0.50,0.00}{##1}}}
\expandafter\def\csname PY@tok@nd\endcsname{\def\PY@tc##1{\textcolor[rgb]{0.67,0.13,1.00}{##1}}}
\expandafter\def\csname PY@tok@s\endcsname{\def\PY@tc##1{\textcolor[rgb]{0.73,0.13,0.13}{##1}}}
\expandafter\def\csname PY@tok@sd\endcsname{\let\PY@it=\textit\def\PY@tc##1{\textcolor[rgb]{0.73,0.13,0.13}{##1}}}
\expandafter\def\csname PY@tok@si\endcsname{\let\PY@bf=\textbf\def\PY@tc##1{\textcolor[rgb]{0.73,0.40,0.53}{##1}}}
\expandafter\def\csname PY@tok@se\endcsname{\let\PY@bf=\textbf\def\PY@tc##1{\textcolor[rgb]{0.73,0.40,0.13}{##1}}}
\expandafter\def\csname PY@tok@sr\endcsname{\def\PY@tc##1{\textcolor[rgb]{0.73,0.40,0.53}{##1}}}
\expandafter\def\csname PY@tok@ss\endcsname{\def\PY@tc##1{\textcolor[rgb]{0.10,0.09,0.49}{##1}}}
\expandafter\def\csname PY@tok@sx\endcsname{\def\PY@tc##1{\textcolor[rgb]{0.00,0.50,0.00}{##1}}}
\expandafter\def\csname PY@tok@m\endcsname{\def\PY@tc##1{\textcolor[rgb]{0.40,0.40,0.40}{##1}}}
\expandafter\def\csname PY@tok@gh\endcsname{\let\PY@bf=\textbf\def\PY@tc##1{\textcolor[rgb]{0.00,0.00,0.50}{##1}}}
\expandafter\def\csname PY@tok@gu\endcsname{\let\PY@bf=\textbf\def\PY@tc##1{\textcolor[rgb]{0.50,0.00,0.50}{##1}}}
\expandafter\def\csname PY@tok@gd\endcsname{\def\PY@tc##1{\textcolor[rgb]{0.63,0.00,0.00}{##1}}}
\expandafter\def\csname PY@tok@gi\endcsname{\def\PY@tc##1{\textcolor[rgb]{0.00,0.63,0.00}{##1}}}
\expandafter\def\csname PY@tok@gr\endcsname{\def\PY@tc##1{\textcolor[rgb]{1.00,0.00,0.00}{##1}}}
\expandafter\def\csname PY@tok@ge\endcsname{\let\PY@it=\textit}
\expandafter\def\csname PY@tok@gs\endcsname{\let\PY@bf=\textbf}
\expandafter\def\csname PY@tok@gp\endcsname{\let\PY@bf=\textbf\def\PY@tc##1{\textcolor[rgb]{0.00,0.00,0.50}{##1}}}
\expandafter\def\csname PY@tok@go\endcsname{\def\PY@tc##1{\textcolor[rgb]{0.53,0.53,0.53}{##1}}}
\expandafter\def\csname PY@tok@gt\endcsname{\def\PY@tc##1{\textcolor[rgb]{0.00,0.27,0.87}{##1}}}
\expandafter\def\csname PY@tok@err\endcsname{\def\PY@bc##1{\setlength{\fboxsep}{0pt}\fcolorbox[rgb]{1.00,0.00,0.00}{1,1,1}{\strut ##1}}}
\expandafter\def\csname PY@tok@kc\endcsname{\let\PY@bf=\textbf\def\PY@tc##1{\textcolor[rgb]{0.00,0.50,0.00}{##1}}}
\expandafter\def\csname PY@tok@kd\endcsname{\let\PY@bf=\textbf\def\PY@tc##1{\textcolor[rgb]{0.00,0.50,0.00}{##1}}}
\expandafter\def\csname PY@tok@kn\endcsname{\let\PY@bf=\textbf\def\PY@tc##1{\textcolor[rgb]{0.00,0.50,0.00}{##1}}}
\expandafter\def\csname PY@tok@kr\endcsname{\let\PY@bf=\textbf\def\PY@tc##1{\textcolor[rgb]{0.00,0.50,0.00}{##1}}}
\expandafter\def\csname PY@tok@bp\endcsname{\def\PY@tc##1{\textcolor[rgb]{0.00,0.50,0.00}{##1}}}
\expandafter\def\csname PY@tok@fm\endcsname{\def\PY@tc##1{\textcolor[rgb]{0.00,0.00,1.00}{##1}}}
\expandafter\def\csname PY@tok@vc\endcsname{\def\PY@tc##1{\textcolor[rgb]{0.10,0.09,0.49}{##1}}}
\expandafter\def\csname PY@tok@vg\endcsname{\def\PY@tc##1{\textcolor[rgb]{0.10,0.09,0.49}{##1}}}
\expandafter\def\csname PY@tok@vi\endcsname{\def\PY@tc##1{\textcolor[rgb]{0.10,0.09,0.49}{##1}}}
\expandafter\def\csname PY@tok@vm\endcsname{\def\PY@tc##1{\textcolor[rgb]{0.10,0.09,0.49}{##1}}}
\expandafter\def\csname PY@tok@sa\endcsname{\def\PY@tc##1{\textcolor[rgb]{0.73,0.13,0.13}{##1}}}
\expandafter\def\csname PY@tok@sb\endcsname{\def\PY@tc##1{\textcolor[rgb]{0.73,0.13,0.13}{##1}}}
\expandafter\def\csname PY@tok@sc\endcsname{\def\PY@tc##1{\textcolor[rgb]{0.73,0.13,0.13}{##1}}}
\expandafter\def\csname PY@tok@dl\endcsname{\def\PY@tc##1{\textcolor[rgb]{0.73,0.13,0.13}{##1}}}
\expandafter\def\csname PY@tok@s2\endcsname{\def\PY@tc##1{\textcolor[rgb]{0.73,0.13,0.13}{##1}}}
\expandafter\def\csname PY@tok@sh\endcsname{\def\PY@tc##1{\textcolor[rgb]{0.73,0.13,0.13}{##1}}}
\expandafter\def\csname PY@tok@s1\endcsname{\def\PY@tc##1{\textcolor[rgb]{0.73,0.13,0.13}{##1}}}
\expandafter\def\csname PY@tok@mb\endcsname{\def\PY@tc##1{\textcolor[rgb]{0.40,0.40,0.40}{##1}}}
\expandafter\def\csname PY@tok@mf\endcsname{\def\PY@tc##1{\textcolor[rgb]{0.40,0.40,0.40}{##1}}}
\expandafter\def\csname PY@tok@mh\endcsname{\def\PY@tc##1{\textcolor[rgb]{0.40,0.40,0.40}{##1}}}
\expandafter\def\csname PY@tok@mi\endcsname{\def\PY@tc##1{\textcolor[rgb]{0.40,0.40,0.40}{##1}}}
\expandafter\def\csname PY@tok@il\endcsname{\def\PY@tc##1{\textcolor[rgb]{0.40,0.40,0.40}{##1}}}
\expandafter\def\csname PY@tok@mo\endcsname{\def\PY@tc##1{\textcolor[rgb]{0.40,0.40,0.40}{##1}}}
\expandafter\def\csname PY@tok@ch\endcsname{\let\PY@it=\textit\def\PY@tc##1{\textcolor[rgb]{0.25,0.50,0.50}{##1}}}
\expandafter\def\csname PY@tok@cm\endcsname{\let\PY@it=\textit\def\PY@tc##1{\textcolor[rgb]{0.25,0.50,0.50}{##1}}}
\expandafter\def\csname PY@tok@cpf\endcsname{\let\PY@it=\textit\def\PY@tc##1{\textcolor[rgb]{0.25,0.50,0.50}{##1}}}
\expandafter\def\csname PY@tok@c1\endcsname{\let\PY@it=\textit\def\PY@tc##1{\textcolor[rgb]{0.25,0.50,0.50}{##1}}}
\expandafter\def\csname PY@tok@cs\endcsname{\let\PY@it=\textit\def\PY@tc##1{\textcolor[rgb]{0.25,0.50,0.50}{##1}}}

\def\PYZbs{\char`\\}
\def\PYZus{\char`\_}
\def\PYZob{\char`\{}
\def\PYZcb{\char`\}}
\def\PYZca{\char`\^}
\def\PYZam{\char`\&}
\def\PYZlt{\char`\<}
\def\PYZgt{\char`\>}
\def\PYZsh{\char`\#}
\def\PYZpc{\char`\%}
\def\PYZdl{\char`\$}
\def\PYZhy{\char`\-}
\def\PYZsq{\char`\'}
\def\PYZdq{\char`\"}
\def\PYZti{\char`\~}
% for compatibility with earlier versions
\def\PYZat{@}
\def\PYZlb{[}
\def\PYZrb{]}
\makeatother


    % Exact colors from NB
    \definecolor{incolor}{rgb}{0.0, 0.0, 0.5}
    \definecolor{outcolor}{rgb}{0.545, 0.0, 0.0}



    
    % Prevent overflowing lines due to hard-to-break entities
    \sloppy 
    % Setup hyperref package
    \hypersetup{
      breaklinks=true,  % so long urls are correctly broken across lines
      colorlinks=true,
      urlcolor=urlcolor,
      linkcolor=linkcolor,
      citecolor=citecolor,
      }
    % Slightly bigger margins than the latex defaults
    
    \geometry{verbose,tmargin=1in,bmargin=1in,lmargin=1in,rmargin=1in}
    
    

    \begin{document}
    
    
    \maketitle
    
    

    
    \hypertarget{sahba-salarian}{%
\section{Sahba Salarian}\label{sahba-salarian}}

    \hypertarget{introduction}{%
\section{Introduction}\label{introduction}}

    Banking industry evolves at a rapid pace with new technology
innovations, changing consumer dynamics, geopolitical movements and
evolving demographics. Currently we are in the midst of and industrial
revolution in banking industry which new models and technologies will
change the way business is conducted.

Lots of new fintech firms or bank tech firms have erupted in recent
times to target newer groups of consumers who prefer to do things
virtually. Some banks have also developed solutions to target thin-file
customers. For example, technology can help monitor various alternative
sources of information for example on creditworthiness, like ensuring
they are paying rent and utilities on time.

There are many areas in this evolving industry where Machine Learning
can be beneficial. All these potensial areas, require advanced
customer's data analysis with high accuracy and prediction capabilities.
Among these areas are:

\begin{enumerate}
\def\labelenumi{\arabic{enumi}.}
\tightlist
\item
  Manage portfolios with algorithms: AI and machine learning could
  streamline the process for developing a portfolio by assessing a
  customer's goals and risk tolerance to develop an individualized
  portfolio.
\item
  Conduct High-Frequency Trading: Trading has the potential to
  accelerate at a faster pace when AI and machine learning algorithms
  are utilized for trading decisions.
\item
  Detect Frauds And Threats to Financial Systems: Routine checks of risk
  factors that could affect customer information can provide an
  understanding of the potential threats. As a result, the response of
  initial invasion detection can be quicker, and the security at
  financial institutions can flag the unusual behavior for monitoring.
\end{enumerate}

In this project, as a step toward fulfilling these ends, advanced data
analysis tools are used to predict and identify which customers of
Santander bank will make a specific transaction in the future,
irrespective of the amount of money transacted.

The data has been provided by Santander bank as a competition for
kagglers on kaggle website. The dataset has 200k observations and 200
column variables for each customer and a class observation defining
whether the customer makes a transaction or not. The target is to
predict a binary classification response with highest posible accuracy
which will be evaluated by comparing the real data from Santander bank
and the predicted values for a separate test data set.

For this project, logistic regression, Decision tree with gradient
boosting and K-fold cross validation, Augmented Decision tree with
gradient boosting and KNN methods are used and compared. The results
from this project were among the top 10 percent accurate predictions for
the unseen datasets and has been awarded the Bronze medal in Kaggle
website.

    \hypertarget{data-engineering-and-visualization}{%
\section{Data Engineering and
Visualization}\label{data-engineering-and-visualization}}

    Data analysis is performed using Python. as mentioned before the dataset
has 200k observations with 202 columns, an ID-code, target response of
zero/one and 200 features. To prepare the dataset, the ID-codes have
been removed from the dataset, unknown variables are chacked and data
set is adjusted using \emph{Pandas} library.

    \hypertarget{balanced-vs.unbalanced-data-set}{%
\subsection{Balanced vs.~Unbalanced data
set}\label{balanced-vs.unbalanced-data-set}}

    Balanced and unbalanced data set usually behave very differently in
terms of the degree of accuracy in classifications. Unbalanced dataset
usually require more flexible models, to give a better classification
accuracy. To investigate the balance in this dataset, the countplot of
the target values which are considered as response ``y'', are plotted
using the \emph{matplot} library. The demonstrated number of
observations in the two target classes demonstrate and unbalance
dataset, since the number of customers willing to make a future
transaction is considerably lower than those who will not make any.

    \begin{center}
    \adjustimage{max size={0.9\linewidth}{0.9\paperheight}}{Santander Customer Transaction Prediction 01_files/Santander Customer Transaction Prediction 01_15_0.png}
    \end{center}
    { \hspace*{\fill} \\}
    
    \hypertarget{feature-distribution-analysis}{%
\subsection{Feature Distribution
Analysis}\label{feature-distribution-analysis}}

    Distribution of features are also demonstrated as follows for the first
and second hundred features, figures\ldots{}..and \ldots{} respectively.
It can be observed that considerable number of features have significant
different distribution for the two target values of 0 and 1. For
example, var\_0, var\_1, var\_2, var\_5, var\_9, var\_13, var\_106,
var\_109, var\_139 and many others. Also some features, like var\_2,
var\_13, var\_26, var\_55, var\_175, var\_184, var\_196 show
distributions resembling the bivariate distribution.

    
    \begin{verbatim}
<Figure size 432x288 with 0 Axes>
    \end{verbatim}

    
    \begin{center}
    \adjustimage{max size={0.9\linewidth}{0.9\paperheight}}{Santander Customer Transaction Prediction 01_files/Santander Customer Transaction Prediction 01_19_1.png}
    \end{center}
    { \hspace*{\fill} \\}
    
    
    \begin{verbatim}
<Figure size 432x288 with 0 Axes>
    \end{verbatim}

    
    \begin{center}
    \adjustimage{max size={0.9\linewidth}{0.9\paperheight}}{Santander Customer Transaction Prediction 01_files/Santander Customer Transaction Prediction 01_20_1.png}
    \end{center}
    { \hspace*{\fill} \\}
    
    \hypertarget{feature-correlation-analysis}{%
\subsection{Feature Correlation
Analysis}\label{feature-correlation-analysis}}
Correlation between the features is an important factor affecting the accuracy of the fits. Since there are 200 features, correlation matrix plots cannot be a good demonstration. To investigate the correlation for this project, the ten most correlated and the ten least correlated features are shown as follows to give a thorough underestanding of the corrlation range. 
\begin{Verbatim}[commandchars=\\\{\}]
{\color{outcolor}Out[{\color{outcolor}44}]:}       level\_0  level\_1         0
         9890  var\_193  var\_172  0.008163
         9891  var\_172  var\_193  0.008163
         9892  var\_162  var\_127  0.008555
         9893  var\_127  var\_162  0.008555
         9894  var\_122  var\_132  0.008956
         9895  var\_132  var\_122  0.008956
         9896  var\_146  var\_169  0.009071
         9897  var\_169  var\_146  0.009071
         9898  var\_183  var\_189  0.009359
         9899  var\_189  var\_183  0.009359
\end{Verbatim}
            
\begin{Verbatim}[commandchars=\\\{\}]
{\color{outcolor}Out[{\color{outcolor}238}]:}    level\_0  level\_1             0
          0  var\_109  var\_126  1.313947e-07
          1  var\_126  var\_109  1.313947e-07
          2  var\_177  var\_100  3.116544e-07
          3  var\_100  var\_177  3.116544e-07
          4  var\_150  var\_116  6.628008e-07
          5  var\_116  var\_150  6.628008e-07
          6  var\_173  var\_176  1.318335e-06
          7  var\_176  var\_173  1.318335e-06
          8  var\_109  var\_157  2.494615e-06
          9  var\_157  var\_109  2.494615e-06
\end{Verbatim}
            
    It can be observed that the correlation between the features are very
small and negligible.

    \hypertarget{pca}{%
\subsection{PCA}\label{pca}}

    To find a low-dimensional representation of the data that captures as
much of the information as possible, PCA analysis is performed. PCA is
sensitive to standardization so the features are standardized before PCA
is performed. In PCA we are interested in the components that maximize
the variance.

    \begin{center}
    \adjustimage{max size={0.9\linewidth}{0.9\paperheight}}{Santander Customer Transaction Prediction 01_files/Santander Customer Transaction Prediction 01_33_0.png}
    \end{center}
    { \hspace*{\fill} \\}
    
    This figure shows that when PC1 has higher values the number of Zeros
increases. It also shows that a radial kernel might be helpful in
finding a linear decision baoundary.

    \hypertarget{traintest-split}{%
\subsection{Train/Test Split}\label{traintest-split}}

    The dataset is randomly splitted into two categories of train and test
sets. Test set is considered to have 33 percent of the whole dataset,
66'000 random observations of the original 200'000.

\begin{Verbatim}[commandchars=\\\{\}]
{\color{outcolor}Out[{\color{outcolor}240}]:} (134000, 200)
\end{Verbatim}
            
    \hypertarget{logistic-regression}{%
\section{Logistic regression}\label{logistic-regression}}

    Logistic regresssion fit is applied on the train set.

\begin{Verbatim}[commandchars=\\\{\}]
{\color{outcolor}Out[{\color{outcolor}241}]:} LogisticRegression(C=1.0, class\_weight=None, dual=False, fit\_intercept=True,
                    intercept\_scaling=1, max\_iter=3000, multi\_class='warn',
                    n\_jobs=None, penalty='l2', random\_state=0, solver='lbfgs',
                    tol=0.01, verbose=0, warm\_start=False)
\end{Verbatim}
            
    Confusion matrix is shown as follows, with TP=1799, FP= 846, TN= 58455,
FN= 4900.

    \begin{center}
    \adjustimage{max size={0.9\linewidth}{0.9\paperheight}}{Santander Customer Transaction Prediction 01_files/Santander Customer Transaction Prediction 01_45_0.png}
    \end{center}
    { \hspace*{\fill} \\}
    
    \hypertarget{roc-curve}{%
\subsection{ROC Curve}\label{roc-curve}}

    ROC curves are appropriate when the observations are balanced between
each class, whereas precision-recall curves are appropriate for
imbalanced datasets. ROC curve is a plot of the false positive rate
(x-axis) versus the true positive rate (y-axis) for a number of
different candidate threshold values between 0.0 and 1.0. The true
positive rate is calculated as the number of true positives divided by
the sum of the number of true positives and the number of false
negatives. It describes how good the model is at predicting the positive
class when the actual outcome is positive. (Sensitivity)True Positive
Rate = True Positives / (True Positives + False Negatives) The false
positive rate is calculated as the number of false positives divided by
the sum of the number of false positives and the number of true
negatives. (Specificity) = True Negatives / (True Negatives + False
Positives)

    \begin{center}
    \adjustimage{max size={0.9\linewidth}{0.9\paperheight}}{Santander Customer Transaction Prediction 01_files/Santander Customer Transaction Prediction 01_50_0.png}
    \end{center}
    { \hspace*{\fill} \\}
    
    \hypertarget{percision-recall-curve}{%
\subsection{Percision-Recall curve}\label{percision-recall-curve}}

    Precision-Recall curves are used for this study, since the dataset has
large class imbalance.

Precision = True Positives / (True Positives + False Positives) Recall =
True Positives / (True Positives + False Negatives) A precision-recall
curve is a plot of the precision (y-axis) and the recall (x-axis) for
different thresholds. The Percision-Recall curve for the logistic model
is as follows:

    \begin{Verbatim}[commandchars=\\\{\}]
f1=0.385 auc=0.502 ap=0.502

    \end{Verbatim}

    \begin{center}
    \adjustimage{max size={0.9\linewidth}{0.9\paperheight}}{Santander Customer Transaction Prediction 01_files/Santander Customer Transaction Prediction 01_54_1.png}
    \end{center}
    { \hspace*{\fill} \\}
    
    It can be observed that although the logistic model has an acceptable
accuracy in calssification of the zero values it is not doing well in
calssification of the unbalanced ``one'' values.


    % Add a bibliography block to the postdoc
    
    
    
    \end{document}
